\section{Geometry}

\subsection{Пересечение прямых}

\begin{equation*}
	AB \coloneqq A - B; CD \coloneqq C - D
\end{equation*}
\begin{equation*}
	(A \times B \cdot CD.x - C \times D \cdot AB.x : A \times B \cdot CD.y - C \times D \cdot AB.y : AB \times CD)
\end{equation*}

\subsection{Касательные}

Точки пересечения общих касательных окружностей с центрами в $(0, 0)$ и $(x, 0)$ равны $\frac{x r_1}{r1 \pm r2}$.
$x$ координата точек касания из $(x, 0)$ равна $\frac{r^2}{x}$.

\subsection{Пересечение полуплоскостей}

Точно так же, как в выпуклой оболочке, но надо добавить bounding box (квадратичного размера относительно координат на входе) и завернуть два раза.
Ответ можно найти как подотрезок от первой полуплоскости типа true до нее же самой на втором круге.
Проверку на вырожденность лучше делать простой проверкой пары-тройки точек из предполагаемого ответа. Стоит быть аккуратнее с точностью.

\inputminted{cpp}{code/halfplanes.cpp}

