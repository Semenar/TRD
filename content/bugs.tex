\section{Bugs}

\begin{itemize}
	\item \mintinline{cpp}{powmod} :)

	\item Всегда чекать Куна дважды, особенно на количество итераций

	\item \mintinline{cpp}{uniform_int_distribution} от одного параметра

	\item \mintinline{cpp}{for (char c : "NEWS")}

	\item Порядок верхних и нижних границ в случае, когда задача двумерна
		$t - b \neq b - t$

	\item \mintinline{cpp}{static} с мультитестами

	\item \mintinline{cpp}{set} со своим компаратором склеивает элементы

	\item Два вектора с соответствующими элементами, сортим один, а элементы
		второго ссылаются на чушь. Предлагается лечить заведением структуры с
		компаратором на каждый чих. В целом, для этого можно написать навороченную
		хрень на шаблонах.

	\item В графе с вершинами степени не больше одного надо писать выделение
		цикла полностью, срезать угол на какой-нибудь тупой меморизации, потому
		что кажется, что он может выглядеть только одним или несколькими
		какими-нибудь специальными способами, не получится, а дебажить сложно.

	\item Структуры, основанные на указателях, не стоит хранить в векторах.

	\item В Карасе для того, чтобы перейти в подстроку, надо сначала идти в
		родителя, а только потом по суфф. ссылкам, эти вещи не коммутируют.

	\item Когда ходим большим количеством указателей по циклу, на единицу
		сдвигается только первый указатель, а остальные могут сдвинуться на много.

	\item \mintinline{cpp}{string str1(str2, 'x'); str1 = 'a';}
\end{itemize}
